\chapter{Topological Spaces and Continuous Functions}
\label{ch2}

\section{Topological Spaces}
\label{ch2sec1}

\section{Basis for a Topology}

\begin{exercise}
\item
  Let \(X\) be a topological space; let \(A\) be a subset of \(X\).
  Suppose that for each \(x\in A\) there is an open set \(U\) contianing \(x\)
  such that \(U\subset A\).
  Show \(A\) is open in \(X\).
  \par\smallskip
  For each \(x\) in \(A\), there exists an open set \(U_x\) such that
  \(x\in U_x\subset A\).
  Then \(\bigcup_{x\in A}U_x\subseteq A\).
  Since \(U_x\) is an open set containing all \(x\in A\),
  \(A = \bigcup_{x\in A}U_x\); that is, \(A\) is the arbitrary union of a
  collection of open set so \(A\) is open in \(X\).
\item
  Consider the nine topologies on the set \(X = \{a,b,c\}\) indicated in
  examaple \(1\) of \cref{ch2}~\cref{ch2sec1}.
  Compare them; that is, for each pair of topologies, determine whether they
  are comparable, and if so, which is finer.
  \par\smallskip
  Let \(X = \{a,b,c\}\).
  Next, let's index the nine topologies.
  \begin{align*}
    \Tau_1 & = \{\varnothing, X\} & \qquad
    \Tau_2 & = \{\varnothing, \{a\}, \{a, b\}, X\} & \qquad
    \Tau_3 & = \{\varnothing, \{b\}, \{a, b\}, \{b, c\}, X\}\\
    \Tau_4 & = \{\varnothing, \{b\}, X\} & \qquad
    \Tau_5 & = \{\varnothing, \{a\}, \{b, c\}, X\} & \qquad
    \Tau_6 & = \{\varnothing, \{b\}, \{c\}, \{a, b\}, \{b, c\}, X\}\\
    \Tau_7 & = \{\varnothing, \{a, b\}, X\} & \qquad
    \Tau_8 & = \{\varnothing, \{a\}, \{b\}, \{a, b\}, X\} & \qquad
    \Tau_9
    & = \{\varnothing, \{a\}, \{b\}, \{c\}, \{a, b\}, \{b, c\}, \{a, c\}, X\}
  \end{align*}
  A topology \(\Tau_i\) is comparable with \(\Tau_j\) for \(i\neq j\) if either
  \(\Tau_i\supset\Tau_j\) or \(\Tau_i\subset\Tau_j\).
  Now \(\Tau_1\subset\Tau_j\) for \(i\in[2,9]\).
  Therefore, \(\Tau_1\) is coarser than \(\Tau_i\) so \(\Tau_i\) is finer than
  \(\Tau_1\).
  We will next look at \(\Tau_9\).
  For \(i\in[1,8]\), \(\Tau_i\subset\Tau_9\).
  Therefore, \(\Tau_9\) is finer than \(\Tau_i\) for \(i\in[1,8]\).
  \(\Tau_2\) is comparable with \(\Tau_7\) and \(\Tau_8\)
  For \(\Tau_8\), \(\Tau_2\subset\Tau_8\) so \(\Tau_8\) is finer than
  \(\Tau_2\), but \(\Tau_7\subset\Tau_2\) so \(\Tau_2\) is finer than
  \(\Tau_7\).
  \(\Tau_3\) is comparable with \(\Tau_4\), \(\Tau_6\), and \(\Tau_7\).
  For \(i = 4,7\), \(\Tau_i\subset\Tau_3\) so \(\Tau_3\) is finer than
  \(\Tau_i\) for \(i = 4,7\), but \(\Tau_3\subset\Tau_6\) so \(\Tau_6\) is
  finer \(\Tau_3\).
  \(\Tau_4\) is comparable to \(\Tau_6\) and \(\Tau_8\) and
  \(\Tau_4\subset\Tau_i\) for \(i = 6,8\) so \(\Tau_i\) is finer than
  \(\Tau_4\).
  \(\Tau_5\) and \(\Tau_6\) are only comparable to \(\Tau_9\) and we have
  already determined the comparability of \(\Tau_9\).
  \(\Tau_7\) is comparable to \(\Tau_8\) and \(\Tau_7\subset\Tau_8\) so
  \(\Tau_8\) is finer than \(\Tau_7\).
  \(\Tau_8\) has been compared with all possible topologies by now.
\item
  Show that the collection \(\Tau_c\) given in example \(4\) of
  \cref{ch2}~\cref{ch2sec1} is a topology on the set \(X\).
  Is the collection
  \[
  \Tau_{\infty} =
  \{U\mid X - U\text{ is infinite or empty or all of } X\}
  \]
  a topology on \(X\)?
\item
  \begin{exercise}[label = (\alph*)]
  \item
    If \(\Tau_{\alpha}\) is a family of topologies on \(X\), show that
    \(\bigcap\Tau_{\alpha}\) is a topology on \(X\).
    Is \(\bigcup\Tau_{\alpha}\) a topology on \(X\)?
  \item
    Let \(\{\Tau_{\alpha}\}\) be a family of topologies on \(X\).
    Show that there is a unique smallest topology on \(X\) containing all the
    collections \(\Tau_{\alpha}\), and a unique largest topology
    contained in all \(\Tau_{\alpha}\).
  \item
    If \(X = \{a,b,c\}\), let
    \[
    \Tau_1 = \{\varnothing, X, \{a\}, \{a, b\}\}\qquad\text{and}\qquad
    \Tau_2 = \{\varnothing, X, \{a\}, \{b, c\}\}.
    \]
    Find the smallest topology containing \(\Tau_1\) and \(\Tau_2\), and the
    largest topology containing \(\Tau_1\) and \(\Tau_2\).
  \end{exercise}
\item
  Show that if \(\mathcal{A}\) is a basis for a topology on \(X\), then the
  topology generated by \(\mathcal{A}\) equals the intersection of all
  topologies on \(X\) that contain \(\mathcal{A}\).
  Prove the same if \(\mathcal{A}\) is a subbasis.
\item
  Show that the topologies on \(\mathbb{R}_{\ell}\) and \(\mathbb{R}_K\) are
  not comparable.
\item
  Consider the following topologies on \(\mathbb{R}\):
  \begin{align*}
    \Tau_1 & = \text{the standard topology,}\\
    \Tau_2 & = \text{the topology of } \mathbb{R}_K,\\
    \Tau_3 & = \text{the finite complement topology,}\\
    \Tau_4 & = \text{the upper limit topology, having all sets } (a, b]
             \text{ as a basis,}\\
    \Tau_5 & = \text{the topology having all sets }
             (-\infty, a) = \{x\mid x < a\} \text{ as a basis.}
  \end{align*}
  Determine, for each of these topologies, which of the others it contains.
\item
  \begin{exercise}[label = (\alph*)]
  \item
    Apply Lemma \(13.2\) to show that the countable collection
    \[
    \mathcal{B} = \{(a, b)\mid a < b, \ a,b\in\mathbb{Q}\}
    \]
    is a basis that generates the standard topology on \(\mathbb{R}\).
  \item
    Show that the collection
    \[
    \mathcal{C} = \{[a, b)\mid a < b, \ a,b\in\mathbb{Q}\}
    \]
    is a basis that generates a topology different from the lower limit
    topology on \(\mathbb{R}\).
  \end{exercise}
\end{exercise}

%%% Local Variables:
%%% mode: latex
%%% TeX-master: t
%%% End:

\chapter{Topological Spaces and Continuous Functions}
\label{ch2}

\section{Topological Spaces}
\label{ch2sec1}

\section{Basis for a Topology}

\begin{exercise}
\item
  Let \(X\) be a topological space; let \(A\) be a subset of \(X\).
  Suppose that for each \(x\in A\) there is an open set \(U\) contianing \(x\)
  such that \(U\subset A\).
  Show \(A\) is open in \(X\).
  \par\smallskip
  For each \(x\) in \(A\), there exists an open set \(U_x\) such that
  \(x\in U_x\subset A\).
  Then \(\bigcup_{x\in A}U_x\subseteq A\).
  Since \(U_x\) is an open set containing all \(x\in A\),
  \(A = \bigcup_{x\in A}U_x\); that is, \(A\) is the arbitrary union of a
  collection of open set so \(A\) is open in \(X\).
\item
  Consider the nine topologies on the set \(X = \{a,b,c\}\) indicated in
  examaple \(1\) of \cref{ch2}~\cref{ch2sec1}.
  Compare them; that is, for each pair of topologies, determine whether they
  are comparable, and if so, which is finer.
  \par\smallskip
  Let \(X = \{a,b,c\}\).
  Next, let's index the nine topologies.
  \begin{align*}
    \Tau_1 & = \{\varnothing, X\}
    & \qquad \Tau_2 & = \{\varnothing, \{a\}, \{a, b\}, X\}
    & \qquad \Tau_3 & = \{\varnothing, \{b\}, \{a, b\}, \{b, c\}, X\}\\
    \Tau_4 & = \{\varnothing, \{b\}, X\}
    & \qquad \Tau_5 & = \{\varnothing, \{a\}, \{b, c\}, X\}
    & \qquad\Tau_6 & = \{\varnothing, \{b\}, \{c\}, \{a, b\}, \{b, c\}, X\}\\
    \Tau_7 & = \{\varnothing, \{a, b\}, X\}
    & \qquad \Tau_8 & = \{\varnothing, \{a\}, \{b\}, \{a, b\}, X\}
    & \qquad \Tau_9 & = \{\varnothing, \{a\}, \{b\}, \{c\}, \{a, b\},
                      \{b, c\}, \{a, c\}, X\}
  \end{align*}
  A topology \(\Tau_i\) is comparable with \(\Tau_j\) for \(i\neq j\) if either
  \(\Tau_i\supset\Tau_j\) or \(\Tau_i\subset\Tau_j\).
  Now \(\Tau_1\subset\Tau_j\) for \(i\in[2,9]\).
  Therefore, \(\Tau_1\) is coarser than \(\Tau_i\) so \(\Tau_i\) is finer than
  \(\Tau_1\).
  We will next look at \(\Tau_9\).
  For \(i\in[1,8]\), \(\Tau_i\subset\Tau_9\).
  Therefore, \(\Tau_9\) is finer than \(\Tau_i\) for \(i\in[1,8]\).
  \(\Tau_2\) is comparable with \(\Tau_7\) and \(\Tau_8\)
  For \(\Tau_8\), \(\Tau_2\subset\Tau_8\) so \(\Tau_8\) is finer than
  \(\Tau_2\), but \(\Tau_7\subset\Tau_2\) so \(\Tau_2\) is finer than
  \(\Tau_7\).
  \(\Tau_3\) is comparable with \(\Tau_4\), \(\Tau_6\), and \(\Tau_7\).
  For \(i = 4,7\), \(\Tau_i\subset\Tau_3\) so \(\Tau_3\) is finer than
  \(\Tau_i\) for \(i = 4,7\), but \(\Tau_3\subset\Tau_6\) so \(\Tau_6\) is
  finer \(\Tau_3\).
  \(\Tau_4\) is comparable to \(\Tau_6\) and \(\Tau_8\) and
  \(\Tau_4\subset\Tau_i\) for \(i = 6,8\) so \(\Tau_i\) is finer than
  \(\Tau_4\).
  \(\Tau_5\) and \(\Tau_6\) are only comparable to \(\Tau_9\) and we have
  already determined the comparability of \(\Tau_9\).
  \(\Tau_7\) is comparable to \(\Tau_8\) and \(\Tau_7\subset\Tau_8\) so
  \(\Tau_8\) is finer than \(\Tau_7\).
  \(\Tau_8\) has been compared with all possible topologies by now.
\item
  Show that the collection \(\Tau_c\) given in example \(4\) of
  \cref{ch2}~\cref{ch2sec1} is a topology on the set \(X\).
  Is the collection
  \[
  \Tau_{\infty} =
  \{U\mid X - U\text{ is infinite or empty or all of } X\}
  \]
  a topology on \(X\)?
  \par\smallskip
  Example \(4\) states: Let \(X\) be a set; let \(\Tau_c\) be the collection of
  all subsets \(U\) of \(X\) such that \(X - U\) is either countable or is all
  of \(X\).
  Then \(\Tau_c\) is a topology on \(X\).
  \par\smallskip
  First, we need to determine if \(X,\varnothing\in\Tau_c\).
  We have that \(X - X = \varnothing\), and since \(\varnothing\) is finite,
  \(\varnothing\) is countable so \(X - X = \varnothing\in\Tau_c\).
  Now, \(X - \varnothing = X\) which is all of \(X\) so
  \(X - \varnothing = X\in\Tau_c\).
  Let \(\{U_{\alpha}\}\) be an indexed family of nonempty elements in
  \(\Tau_c\).
  Then
  \[
  X - \bigcup_{\alpha}U_{\alpha} = \bigcap_{\alpha}(X - U_{\alpha})
  \]
  By definition of the problem, \(U_{\alpha}\) is such that \(X - U_{\alpha}\)
  is either countable or all of \(X\).
  Since \(X - U_{\alpha}\) is countable, \(\bigcap_{\alpha}(X - U_{\alpha})\).
  Therefore \(\bigcup_{\alpha}U_{\alpha}\in\Tau_c\).
  Now, let's take a finite intersections of elements of \(\Tau_c\).
  Then
  \[
  X - \bigcap_{i = 1}^nU_i = \bigcup_{i = 1}^n(X - U_i)
  \]
  Agian, we have that \(X - U_i\) is countable and the finite union of
  countable sets is countable so \(\bigcap_1^nU_i\in\Tau_c\) and \(\Tau_c\) is
  a topology on \(X\).
  For the second part, let \(X = \mathbb{R}\).
  Let \(U_1 = \mathbb{R}\setminus\{x\}\) where \(x\) can be any real number you
  choose.
  Then \(X - U_1 = \{x\}\) which is a single element and is hence finite.
  Therefore, \(\Tau_{\infty}\) is not a topology on \(X\).
\item
  \begin{exercise}[label = (\alph*)]
  \item
    If \(\Tau_{\alpha}\) is a family of topologies on \(X\), show that
    \(\bigcap\Tau_{\alpha}\) is a topology on \(X\).
    Is \(\bigcup\Tau_{\alpha}\) a topology on \(X\)?
    \par\smallskip
    Since, for each \(\alpha\), \(\Tau_{\alpha}\) is a topology,
    \(X,\varnothing\in\Tau_{\alpha}\) so
    \(X,\varnothing\in\bigcap\Tau_{\alpha}\).
    Let \(U_n\in\bigcap\Tau_{\alpha}\) be basis elements of
    \(\bigcap\Tau_{\alpha}\).
    Then \(U_n\), for each \(\alpha\), exists in \(\Tau_{\alpha}\).
    Therefore, \(U_n\) is a basis element for each \(\Tau_{\alpha}\).
    \(\bigcap_1^kU_i\) is the finite intersection of basis elements of
    \(\Tau_{\alpha}\); thus, \(\bigcap_1^kU_i\in\Tau_{\alpha}\).
    Let's consider the arbitrary intersection of \(U_n\).
    For each \(\alpha\), \(U_n\) exist in \(\Tau_{\alpha}\).
    Then \(\bigcup U_n\in\Tau_{\alpha}\).
    Hence \(\bigcap\Tau_{\alpha}\) is a topology on \(X\).
    For the union, let \(X = \{a,b,c\}\) and
    \(\Tau_{\alpha} = \{\Tau_1,\Tau_2\}\) where
    \(\Tau_1 = \{\varnothing,\{a\},X\}\) and
    \(\Tau_2 = \{\varnothing,\{b\},X\}\).
    Then \(\bigcup\Tau_{\alpha} = \{\varnothing,\{a\},\{b\},X\}\).
    Now the union of elements in \(\Tau_{\alpha}\) must be in
    \(\Tau_{\alpha}\).
    However, \(\{a\}\cup\{b\} = \{a,b\}\not\in\Tau_{\alpha}\).
    Hence, the union is not a topology on \(X\).
  \item
    Let \(\{\Tau_{\alpha}\}\) be a family of topologies on \(X\).
    Show that there is a unique smallest topology on \(X\) containing all the
    collections \(\Tau_{\alpha}\), and a unique largest topology
    contained in all \(\Tau_{\alpha}\).
  \item
    If \(X = \{a,b,c\}\), let
    \[
    \Tau_1 = \{\varnothing, X, \{a\}, \{a, b\}\}\qquad\text{and}\qquad
    \Tau_2 = \{\varnothing, X, \{a\}, \{b, c\}\}.
    \]
    Find the smallest topology containing \(\Tau_1\) and \(\Tau_2\), and the
    largest topology contained in \(\Tau_1\) and \(\Tau_2\).
    \par\smallskip
    The smallest topology containing \(\Tau_1\) and \(\Tau_2\) is
    \(\Tau_s = \{\varnothing,\{a,b\},\{b,c\},X\}\) and the largest contained in
    is
    \(\Tau_l = \{\varnothing,\{a\},X\}\).
  \end{exercise}
\item
  Show that if \(\mathcal{A}\) is a basis for a topology on \(X\), then the
  topology generated by \(\mathcal{A}\) equals the intersection of all
  topologies on \(X\) that contain \(\mathcal{A}\).
  Prove the same if \(\mathcal{A}\) is a subbasis.
  \par\smallskip
  Let \(\Tau\) be the be the topology generated by \(\mathcal{A}\) and let
  \(\{\Tau_{\alpha}\}\) be the family of topologies that contain
  \(\mathcal{A}\).
  Since \(\mathcal{A}\subset\cap\Tau_{\alpha}\) and
  \(\mathcal{A}\subset\Tau\), we have that \(\cap\Tau_{\alpha}\subset\Tau\).
  Let \(U\) be an open set in \(\mathcal{A}\).
  Then \(\cup U\in\Tau_{\alpha}\) for all \(\alpha\) but \(\Tau_{\alpha}\)
  contains \(\mathcal{A}\) so \(\Tau\subset\cap\Tau_{\alpha}\).
  Therefore, \(\Tau = \cap\Tau_{\alpha}\).
\item
  Show that the topologies on \(\mathbb{R}_{\ell}\) and \(\mathbb{R}_K\) are
  not comparable.
  \par\smallskip
  The lower limit topology, \(\mathbb{R}_{\ell}\), is
  \([a,b) = \{x\mid a\leq x < b\}\) and the K-topology, \(\mathbb{R}_K\),
  is the collection of all open intervals \((a,b)\) along with all sets of the
  form \((a,b) - K\) where \(K = \{1/n\mid n\in\mathbb{Z}^+\}\).
  Consider the basis element \([2,3)\) in the lower limit topology.
  \([2,3)\) is open in \(\mathbb{R}_{\ell}\), but I claim it isn't open in
  \(\mathbb{R}_K\).
  Let \(\mathcal{B}\) be a basis element for \(\mathbb{R}_K\).
  Suppose \(2\in\mathcal{B}\subset [2,3)\).
  Basis elements in \(\mathbb{R}_K\) are of the form \((a,b)\) or
  \((a,b) - K\).
  If \(2\in\mathcal{B}\), then \(a < 2\).
  Therefore, there exists an \(x\) such that \(\max(a,1) < x < 2\) so
  \(x\not\in K\).
  Then \(x\) is in the basis element \(\mathcal{B}\), but \(x\not\in [2,3)\).
  Thus, \([2,3)\) is not open in \(\mathbb{R}_K\).
  Consider the basis element \((-1,1) - K\).
  Now, \((-1,1) - K\) is open in \(\mathbb{R}_K\), but I claim it isn't open in
  \(\mathbb{R}_{\ell}\).
  Let \(\mathcal{B}\) be a basis element of \(\mathbb{R}_{\ell}\) such that
  \(0\in\mathcal{B}\subset\{(-1,1) - K\}\).
  Let \(\mathcal{B} = [a,b)\) be a basis element containg zero.
  Then \(a\leq 0\) and \(b > 0\).
  Let \(n\) be an integer such that \(nb > 1\iff 1/n < b\).
  Then \(1/n\in K\) and \(1/n\in\mathcal{B}\).
  Therefore, \(1/n\not\in (-1,1)\) so \(\mathcal{B}\not\subset\{(-1,1) - K\}\).
\item
  Consider the following topologies on \(\mathbb{R}\):
  \begin{align*}
    \Tau_1 & = \text{the standard topology, having all sets } (a, b)
             \text{ as a basis,}\\
    \Tau_2 & = \text{the topology of } \mathbb{R}_K,\\
    \Tau_3 & = \text{the finite complement topology, having all sets }
             \Tau = \{X\subset\mathbb{R}\colon X = \varnothing\text{ or }
             \mathbb{R}\setminus X\text{ is finite}\}\\
    \Tau_4 & = \text{the upper limit topology, having all sets } (a, b]
             \text{ as a basis,}\\
    \Tau_5 & = \text{the topology having all sets }
             (-\infty, a) = \{x\mid x < a\} \text{ as a basis.}
  \end{align*}
  Determine, for each of these topologies, which of the others it contains.
  \par\smallskip
  Lemma \(13.4\colon\) The topologies of \(\mathbb{R}_{\ell}\) and
  \(\mathbb{R}_K\) are strictly finer than the standard topology on
  \(\mathbb{R}\), but are not comparbale with one another where
  \(\mathbb{R}_{\ell}\) is the lower limit topology.
  However, from lemma \(13.4\), we can show that upper limit topology is
  strictly finer than the standard topology.
  Therefore, we know that \(\Tau_1\subset\Tau_2,\Tau_4\).
\item
  \begin{exercise}[label = (\alph*)]
  \item
    Apply Lemma \(13.2\) to show that the countable collection
    \[
    \mathcal{B} = \{(a, b)\mid a < b, \ a,b\in\mathbb{Q}\}
    \]
    is a basis that generates the standard topology on \(\mathbb{R}\).
  \item
    Show that the collection
    \[
    \mathcal{C} = \{[a, b)\mid a < b, \ a,b\in\mathbb{Q}\}
    \]
    is a basis that generates a topology different from the lower limit
    topology on \(\mathbb{R}\).
  \end{exercise}
\end{exercise}

\section{The Order Topology}

\section{The Product Topology on \(X\times Y\)}

\section{The Subspace Topology}

\begin{exercise}
\item
  Show that if \(Y\) is a subspace of \(X\), and \(A\) is a subset of \(Y\),
  then the topology \(A\) inherits as a subspace of \(Y\) is the same as the
  topology it inherits as a subspace of \(X\).
  \par\smallskip
  Let \(\Tau_X\) be the topology on \(X\) and \(U\) be a basis of \(\Tau_X\).
  Then \(\Tau_Y = \{Y\cap U\colon U\in\Tau_X\}\).
  Let \(V = Y\cap U\).
  Then \(\Tau_A = \{A\cap V\colon V\in\Tau_Y\}\).
  We now have that \(A\cap V = A\cap Y\cap U\), and since
  \(A\subset Y\subset X\), \(A\cap Y = A\).
  Then \(\Tau_A^X = \{A\cap U\colon U\in \Tau_X\}\) and
  \(\Tau_A\subset\Tau_A^X\).
  Let \(U\cap A\in\Tau_A^X\).
  Since \(U\) is open in \(X\), \(U\cap Y\) is open relative to \(Y\) so
  \(U\cap Y\in\Tau_Y\).
  Then \(U\cap Y\cap A\in\Tau_A\) since \(U\cap Y\in\Tau_Y\) and
  \(U\cap Y\cap A = U\cap A\in\Tau_A\).
  Therefore, \(\Tau_A = \Tau_A^X\) and the topology \(A\) inherits as a
  subspace of \(Y\) is the same as the topology it inherits as a subspace of
  \(X\).
\item
  If \(\Tau\) and \(\Tau'\) are topologies on \(X\) and \(\Tau'\) is strictly
  finer than \(\Tau\), \(\Tau'\supset\Tau\), what can you say about the
  corresponding subspace topologies on the subset \(Y\) of \(X\)?
  \par\smallskip
  The inherited topologies can be the same.
\item
  Consider the set \(Y = [-1,1]\) as a subspace of \(\mathbb{R}\).
  Which of the following sets are open in \(Y\)? Which are open in \
  \(\mathbb{R}\)?
  \begin{align*}
    A & = \{x\colon 1/2 < \lvert x\rvert < 1\}\\
    B & = \{x\colon 1/2 < \lvert x\rvert\leq 1\}\\
    C & = \{x\colon 1/2\leq \lvert x\rvert < 1\}\\
    D & = \{x\colon 1/2\leq \lvert x\rvert\leq 1\}\\
    E & = \{x\colon 0 < \lvert x\rvert < 1\text{ and } 1/x\not\in\mathbb{Z}^+\}
  \end{align*}
\item
  A map \(f\colon X\to Y\) is said to be an \textit{open map} if for every open
  set \(U\) of \(X\), the set \(f(U)\) is open in \(Y\).
  Show that \(\pi_1\colon X\times Y\to X\) and \(\pi_2\colon X\times Y\to Y\)
  are open maps.
\item
  Let \(X\) and \(X'\) denote a single set in the topologies \(\Tau\) and
  \(\Tau'\), respectively; let \(Y\) and \(Y'\) denote a single set in the
  topologies \(U\) and \(U'\), respectively.
  Assume these sets are nonempty.
  \begin{exercise}[label = (\alph*), ref = \arabic{exercisei} (\alph*)]
  \item
    \label{ch2sec55a}
    Show that if \(\Tau'\supset\Tau\) and \(U'\supset U\), then the product
    topology on \(X'\times Y'\) is finer than the product topology on
    \(X\times Y\).
  \item
    Does the converse of \cref{ch2sec55a} hold?
    Justify your answer.
  \end{exercise}
\item
  Show that the countable collection
  \[
  \{(a, b)\times (c, d)\colon a < b\text{ and } c < d, \text{ and } a, b, c, d
  \text{ rational}\}
  \]
  is a basis for \(\mathbb{R}^2\).
\item
  Let \(X\) be an ordered set.
  If \(Y\) is a proper subset of \(X\) that is convex in \(X\), does it follow
  that \(Y\) is an interval or a ray in \(X\)?
\item
  If \(L\) is a straight line in the plane, describe the topology \(L\)
  inherits as a subspace of \(\mathbb{R}_{\ell}\times\mathbb{R}\) and as a
  subspace of \(\mathbb{R}_{\ell}\times\mathbb{R}_{\ell}\).
  In each case it is a familiar topology.
\item
  Show that the dictionary order topology on the set
  \(\mathbb{R}\times\mathbb{R}\) is the same as the product topology
  \(\mathbb{R}_d\times\mathbb{R}\), where \(\mathbb{R}_d\) denotes
  \(\mathbb{R}\) in the discrete topology.
  Compare this topology with the standard topology on \(\mathbb{R}^2\).
\item
  Let \(I = [0,1]\).
  Compare the product topology on \(I\times I\), the dictionary order topology
  on \(I\times I\), and the topology \(I\times I\) inherits as a subspace of
  \(\mathbb{R}\times\mathbb{R}\) in the dictionary order topology.
\end{exercise}

%%% Local Variables:
%%% mode: latex
%%% TeX-master: t
%%% End:
